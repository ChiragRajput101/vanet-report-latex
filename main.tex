\documentclass[11pt]{IEEEphot}

% Add packages for custom formatting
\usepackage{xcolor}
\usepackage{fancyhdr}

\newtheorem{theorem}{Theorem}
\newtheorem{lemma}{Lemma}

% Custom header/footer with black background
\pagestyle{fancy}
\fancyhf{} % Clear default header/footer

% Header
\fancyhead[L]{\colorbox{black}{\parbox{0.95\textwidth}{\color{white}\small Malicious Intent Detection in ITS}}}
\fancyhead[R]{\colorbox{black}{\parbox{0.95\textwidth}{\color{white}\small }}}

% Adjust header/footer size
\setlength{\headheight}{20pt}
\setlength{\footskip}{30pt}

\begin{document}

% Title Section
\title{Malicious Intent Detection\\ in Intelligent Transport Systems}

% Author Section
\author{Pranav Kumar    (2021UCA1910)\\ Rohan Kumar Mishra  (2021UCA1919)\\
Chirag Rajput   (2021UCA1921)}

\affil{Dept. of Computer Science}

% Generate the title and abstract page
\maketitle
\markboth{\color{white}Malicious Intent Detection in ITS}{\color{white}}

% Abstract Section
\begin{abstract}
Your abstract text goes here. Write a concise and informative summary of your report, usually between 150–200 words. Ensure it captures the key objectives, methods, and conclusions of your study. Keep the tone formal and precise.
\end{abstract}

% Keywords Section
\begin{IEEEkeywords}
Keyword1, Keyword2, Keyword3, Keyword4.
\end{IEEEkeywords}

% Sections
\section{Introduction}
\vspace{10pt}
Emerging wireless technologies allow vehicles to connect with each other through wireless channels, forming a vehicular network that enhances safety and driving experience by sharing traffic and entertainment information. Technologies like Bluetooth, Dedicated Short Range Communication (DSRC), cellular networks, and WiFi are being integrated into vehicles by manufacturers. These innovations enable applications such as improved driving safety, smart roadside systems, and environmentally friendly transportation.
\\[0pt]
\\
In a vehicular network, vehicles exchange critical data, such as traffic conditions, road construction, accidents, weather alerts, and road status. Given the importance of this information, it is vital for vehicles and drivers to distinguish between trustworthy and untrustworthy data. Vehicles that share accurate information are deemed trustworthy, while those transmitting false data are seen as untrustworthy. Many research efforts in this area rely on Public Key Infrastructure (PKI) to ensure secure communication. While PKI provides an initial security layer, it cannot prevent legitimate vehicles from sending unreliable information due to faulty sensors, malware, or malicious actions.
\\
Vehicular communication occurs through Vehicle-to-Vehicle (V2V), Vehicle-to-Infrastructure (V2I), or hybrid modes via DSRC, in single or multi-hop configurations. The Intelligent Transport System (ITS) ecosystem includes vehicles with on-board units (OBUs), roadside units (RSUs), and back-end systems for services like registration and authorization. These components collaborate to improve traffic management and road safety.
\\
\\
However, due to the high mobility of vehicles and the dynamic nature of the network, trust between neighboring vehicles is often uncertain. This issue is worsened when malicious vehicles are present, potentially spreading false information. \\
\\
\\
For instance, a malicious vehicle could claim that a road is clear while it is congested, posing significant risks to traffic safety and efficiency. Hence, evaluating vehicle trustworthiness is essential for maintaining the integrity of vehicular networks.
\\
\\
A trust management system is vital for vehicles to assess the trustworthiness of the messages they receive, enabling them to make informed decisions. Additionally, such systems allow network operators to base rewards or punishments on specific vehicles' behaviors. The trust value of a vehicle is usually calculated based on ratings of its past actions, which are generated by neighboring vehicles or relevant network nodes. Trust management systems can be divided into two main types: centralized and decentralized.
\\
Centralized trust management systems involve storing and processing all ratings in a central server, often a cloud-based server. While this structure allows for consistent data processing, it faces challenges in meeting the strict Quality-of-Service (QoS) requirements of vehicular networks, where vehicles often need to make real-time decisions with minimal delay. The latency introduced by centralized systems may not meet the low-latency demands essential in these high-speed, dynamic environments, making them less suitable for applications requiring fast decision-making and response times.
\\
\\
On the other hand, decentralized trust management systems conduct trust management locally within the vehicle or through a Roadside Unit (RSU). This reduces the dependence on centralized network infrastructures and minimizes delays associated with data transmission to a central server. However, the reliability of ratings in decentralized systems can be questionable because a single vehicle may have limited capacity to observe or accurately assess events. The varying conditions under which different vehicles operate further complicate the trust evaluation process. Moreover, the high variability in the network topology, with vehicles constantly moving and interacting with new neighbors, makes it challenging to evaluate the trustworthiness of every vehicle in real-time.
\\
Some trust management models use RSUs to handle trust assessments, but RSUs are vulnerable to malfunctions, intrusions, or security breaches due to their outdoor deployment. This makes them unreliable as a consistent trust service provider across the entire vehicular network. Therefore, effectively managing trust in vehicular networks remains an urgent problem that requires robust and efficient solutions to ensure the system's security and reliability.
\\
With this in mind, blockchain is seen as a promising solution to address the challenges discussed above. Initially recognized as a disruptive technology in the financial industry, blockchain allows distributed nodes to interact and maintain a secure, tamper-proof ledger without relying on a centralized authority like a bank. Due to its high level of security and reliability, blockchain has been explored and applied in various non-financial domains, including content delivery, key management, and decentralized storage.
\\
\\
The decentralized nature of blockchain makes it a suitable tool for trust management in vehicular networks, particularly among distributed Roadside Units (RSUs). By utilizing blockchain, RSUs can collaborate and maintain a consistent and reliable database, effectively mitigating the issues associated with centralized systems. Even if a small fraction of RSUs are compromised by attackers, their ability to generate blocks is much slower compared to that of legitimate RSUs. As a result, blockchain-based systems can efficiently manage trust in vehicular networks, enabling vehicles to assess the trustworthiness of neighboring vehicles and the credibility of the messages they receive.
\section{Motivation}
\vspace{10pt}
The large bandwidth, mobile edge computing, and edge cloud collaboration technologies provided by 5G have significantly accelerated the development of Intelligent Transport Systems (ITS) in recent years. These advancements have enabled faster data transmission and enhanced communication between vehicles, contributing to the growth of intelligent transportation systems. However, this rapid progress has also introduced substantial security risks. In ITS, vehicles exchange a vast number of messages, some of which are critical for ensuring road safety. Unfortunately, not all of these messages can be trusted, as some vehicles may have malicious intentions or faulty sensors that provide inaccurate information.
\\
\\
Security must be prioritized in ITS, as broadcast messages often contain vital information, such as traffic updates, road conditions, and accident alerts, which are crucial for vehicle decision-making. If these messages are tampered with or manipulated, it could result in accidents, traffic congestion, or other disastrous outcomes. Consequently, ensuring the authenticity and reliability of exchanged messages is paramount. A failure to implement proper security measures could have dire consequences, making it essential to incorporate robust trust management mechanisms within ITS.
\\
These mechanisms help establish and evaluate the trustworthiness of entities, such as vehicles and messages, allowing vehicles to make informed decisions based on reliable information.
\\
\\
Current trust management models in ITS, however, face several challenges that need to be addressed. One of the major difficulties arises from the high speed at which vehicles travel, making it challenging to gather sufficient data to calculate a vehicle’s trust value. Moreover, many existing trust-based security schemes rely on complex iterative processes to assess trustworthiness, which can significantly impact network performance and increase latency, particularly in high-speed environments where quick decision-making is required. Additionally, the dynamic nature of the wireless communication environment in ITS exposes vehicle data to potential interception or manipulation by hackers, who could take control of vehicle operations and cause substantial harm.
\\
Finally, ensuring the overall security of the trust management system remains a crucial unresolved problem. To address these challenges, we propose a novel trust management model that offers enhanced security, reduces latency, and improves the reliability of vehicle communications in ITS. This model aims to optimize trust evaluations and ensure the safety and efficiency of intelligent transportation systems in the face of rapidly evolving threats and network dynamics.

\vspace{15pt}
\section{Literature Survey}
\vspace{5pt}
Centralized trust management in vehicular networks has been extensively explored in recent studies. In these approaches [1] [2], a central server is used to collect, compute, and store the trust values of all vehicles. The central server is typically considered a fully trusted entity that is not vulnerable to attacks.
\\
\\
For example, in [1], a reputation-based announcement scheme is introduced for vehicular networks. In this system, vehicles detect traffic-related events and publish announcements to nearby vehicles. The receiving vehicles are then required to assess the credibility of the messages and generate feedback reports. These feedbacks are collected by a centralized reputation server, which uses the data to update the reputation values and issue certificates for all vehicles in the network.
\\
\\
Additionally, M. E. Mahmoud et al. proposed a stimulation and punishment mechanism for mobile nodes in [3]. This mechanism uses a "micropayment" system to incentivize nodes to relay packets from other vehicles. Honest nodes earn credits that they can use when they need to relay data. A reputation system is also integrated to address issues like packet droppers.
\\
Gamage et al.  introduced a centralized trust management framework that employed a unified cloud server to manage decentralized nodes. The model relied on context-sensitive information from individual services and their interaction modes to predict trust values during the design phase. 
\\
Aamir et al. [4] developed a user-position-based trust model that evaluated the credibility of social sensor cloud services by extracting trust ratings through the analysis of their text features. 
\\
Similarly, Zhang et al. [5] applied a deep reinforcement learning approach, utilizing software-defined trust to enhance decision-making in trust management.
\\
While the centralized trust management model can ensure consistency in trust information across different regions via centralized servers, these servers are prone to network attacks and vulnerable to single-point failures. One significant limitation of this model is the reliance on continuous requests for trust data from the server. When the server becomes overloaded, nodes may struggle to access real-time trust data, leading to data loss and increased risks. 
\\
\\
Asuquo et al. [6] introduced a decentralized trust management system designed to detect and eliminate fraudulent nodes. Initially, direct trust was established by combining the number of forwarded evidence with the energy consumption of nodes. Subsequently, recommendation trust was computed using a formula that considered indirect trust, recommendation reliability, and familiarity. 
\\
Babbitt and Szymanski [7] proposed a trust management approach that leveraged erasure coding to enhance the message transfer rate while assessing the credibility of nodes across all paths that deliver message segments to the destination.
\\
\\
Cai et al. [8] introduced the Evolved Self-Cooperative Trust (ESCT) mechanism to mitigate routing disruption attacks. This scheme uses trust level data and cognitive evaluation to assess the trustworthiness of information shared by mobile nodes, allowing nodes to dynamically refine their cognitive assessments and distinguish between trustworthy and malicious information.
\\
\\
Finally, Ahmad et al. [9] proposed a novel trust model capable of efficiently identifying dishonest nodes that launched attacks and revoked their certificates. In this model, each node begins by establishing trust for the sender through a multidimensional credibility assessment. Collectively, these studies emphasize the significance of various decentralized trust management mechanisms in enhancing the security of interconnected systems and fostering trust between nodes.
\\
In contrast to the centralized trust management model, the distributed trust management model offers advantages like reduced processing time and enhanced scalability .
\\
In decentralized trust management systems, the absence of an immutable ledger can lead to significant vulnerabilities. Without an immutable record, malicious nodes could easily manipulate or falsify trust data, potentially altering trust scores or spreading false information across the network. This could undermine the reliability of the entire system, as nodes may be misled into trusting dishonest or compromised entities. 
\\
\\
Additionally, the lack of an immutable ledger makes it difficult to trace the origins of fraudulent behavior, further complicating the identification and removal of malicious actors. In scenarios where trust is critical, such as vehicular networks or intelligent transport systems, the integrity of data must be preserved to ensure secure and reliable interactions between nodes.
\\
\\
To address these challenges, we propose a trust management model that leverages blockchain technology. Blockchain is a distributed ledger system that links data records into blocks, creating a secure and immutable chain using timestamps and cryptographic hashes. 
\\
The decentralized nature of blockchain enables effective trust management among distributed RSUs, reducing the risks associated with single points of failure. With blockchain, multiple RSUs can collaboratively maintain a reputation database, eliminating the need for a centralized server. In conclusion, our proposed blockchain-based trust management model offers a secure and reliable approach to establishing trust in distributed systems.
\\
\\
Recently, many researchers have explored the application of blockchain technology in trust management. Zhang et al. [10] introduced a blockchain-based trust management system for vehicle networks supporting AI. In this system, each vehicle generates and exchanges information with neighboring vehicles, which are then verified by those vehicles. Upon receiving and verifying messages, a vehicle establishes trust for nearby vehicles, using deep learning algorithms to manage this process. 
\\
Yang et al. [11] proposed a decentralized trust management system for vehicular networks based on blockchain technology. In this system, vehicles utilize Bayesian inference models to validate messages from neighboring vehicles. Based on these verification results, each vehicle generates ratings for the message sources, which are then uploaded to RSUs. 
\\
These models overcome the limitations of traditional distributed trust management, offering a consistent, tamper-proof, and decentralized solution for managing trust.





 

% References Section Placeholder
\bibliographystyle{IEEEtran}
\bibliography{references}
1.  Q. Li, A. Malip, K. M. Martin, S. Ng, and J. Zhang, “A reputation-based announcement scheme for VANETs,” IEEE Transactions on Vehicular Technology, vol. 61, no. 9, pp. 4095-4108, Nov. 2012 
\\
2. C. Lai, K. Zhang, N. Cheng, H. Li, and X. Shen, “SIRC: A secure incentive scheme for reliable cooperative downloading in highway VANETs,” IEEE Transactions on Intelligent Transportation Systems, vol. 18, no. 6, pp. 1559-1574, June 2017. 
\\
3. M. Mahmoud and X. Shen, “An integrated stimulation and punishment mechanism for thwarting packet dropping attack in multihop wireless networks,” IEEE Transactions on Vehicular Technology, vol. 60, no. 8, pp. 3947-3962, Oct. 2011. 
\\
4. T. Aamir, D. Hai, and A. Bouguettaya, “Trust in social-sensor cloud service,” in Proc. IEEE Int. Conf. Web Services (ICWS), 2018, pp. 359–362. 
\\
5. D. Zhang, F. R. Yu, and R. Yang, “A machine learning approach for software-defined vehicular ad hoc networks with trust management,” in Proc. IEEE Global Commun. Conf. (GLOBECOM), 2019, pp. 1–6.  
\\
6. P. Asuquo, H. Cruickshank, C. P. A. Ogah, A. Lei, and Z. Sun, “A distributed trust management scheme for data forwarding in satellite DTN emergency communications,” IEEE J. Sel. Areas Commun., vol. 36, no. 2, pp. 246–256, Feb. 2018. 
\\
7.  T. A. Babbitt and B. K. Szymanski, “Trust management in delay tolerant networks utilizing erasure coding,” in Proc. IEEE Int. Conf. Commun. (ICC), 2017, pp. 6331–6337. 
\\
8.  R. J. Cai, X. J. Li, and P. H. J. Chong, “An evolutionary self-cooperative trust scheme against routing disruptions in MANETs,” IEEE Trans. Mobile Comput., vol. 18, no. 1, pp. 42–55, Jan. 2019. 
\\
9.  F. Ahmad, F. Kurugollu, A. Adnane, R. Hussain, and F. Hussain, “MARINE: Man-in-the-middle attack resistant trust model in connected vehicles,” IEEE Internet Things J., vol. 7, no. 4, pp. 3310–3322, Apr. 2020. 
\\
10. C. Zhang, W. Li, Y. Luo, and Y. Hu, “AIT: An AI-enabled trust management system for vehicular networks using blockchain technology,” IEEE Internet Things J., vol. 8, no. 5, pp. 3157–3169, Mar. 2021. 
\\
11. Z. Yang, K. Yang, L. Lei, K. Zheng, and V. C. M. Leung, “Blockchain based decentralized trust management in vehicular networks,” IEEE Internet Things J., vol. 6, no. 2, pp. 1495–1505, Apr. 2019. 
\end{document}
